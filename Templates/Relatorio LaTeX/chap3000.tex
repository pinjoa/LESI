
\chapter{Capitulo 3}

\section{Sub-capitulo 1}
Texto do sub-capitulo 1 que inclui um pouco de código...

\subsection{Lista dos carateres a ignorar}
\begin{Verbatim}
	t_ignore = " \t\n"
\end{Verbatim}
\hfill \break

\section{Sub-capitulo 2}
Texto do sub-capitulo 2

\subsection{Listagem de programa}
\begin{lstlisting}[language={c},
caption={Código do programa: \textbf{apaga}.},
label=lst:p1e]
/**
* programa apaga.c
* @author #20808 Joao Carlos Pinto
*/
#include <stdio.h>
#include <unistd.h>
#include <errno.h>
#include <string.h>
#include "mytools.h"

int main(int argc, char *argv[]) {
  char buffer[MAXBUFFERSIZE];
  if (argc < 2) {
    buffer[0] = '\0';
    sprintf(buffer, "Falta: ficheiro\n Deve "+
    "utilizar-se desta forma:\n%s ficheiro(s)\n", 
    argv[0]);
    escrevErro(buffer);
    return 1;
  }
  int resultado, bkErrno;
  for(int i=1; i<argc; i++){
    if ((resultado = unlink(argv[i])) == -1) {
      buffer[0] = '\0';
      bkErrno = errno;
      sprintf(buffer, "erro (%d, %s) ao apagar o "+
      "ficheiro: %s\n", bkErrno, strerror(bkErrno), 
      argv[i]);
      escrevErro(buffer);
      return 1;
    }
    if (resultado == 0) {
      buffer[0] = '\0';
      sprintf(buffer, "o ficheiro \"%s\" foi apagado!\n", 
      argv[i]);
      escrever(buffer);
    } else {
      buffer[0] = '\0';
      bkErrno = errno;
      sprintf(buffer, "resultado (%d, %s) inesperado ao "+
      "apagar o ficheiro \"%s\"!\n", bkErrno, 
      strerror(bkErrno), argv[i]);
      escrevErro(buffer);
      return 1;
    }
  }
  return 0;
}
\end{lstlisting}
